 \section{目标规划}
 \subsection{目标规划问题及其数学模型}
 目标规划\footnote{目标规划可分为线性和非线性两种,这里讨论的是比较简单的线性目标规划 
(linear goal programming)。} (goal programming) 是运筹学中的一个重要分支,它是为解决多目标决策问题而发展起来
 的一种数学方法。目标规划可以按照确定的若干目标值及其实现的优先次序,在给定约束条件下寻找
 偏离目标值最小的解的数学方法。它在处理实际决策问题时,承认各项决策要求 (即使是冲突的) 
 的存在有合理性;在做最终决策时,不强调其绝对意义上的最优性。由于目标规划在一定程度上弥补了线性规划
 的局限性,因此,目标规划被认为是一种较之线性规划更接近于实际决策工程的工具。


 目标规划数学模型的一般形式为:
 \begin{equation}\label{eq:gp}
\begin{array}{ll}
 \min P_l (\sum\limits_{k = 1}^K ({W_{lk}^- d_k^- + W_{lk}^ +  d_k^+  } ))&l=1,2,\ldots,L \\
 \multicolumn{2}{l}{\text{s.t.}
 \left\{ \begin{array}{ll}
\sum\limits_{j = 1}^n {c_{kj}x_j+ d_k^- -d_k^+ }=g_k            &k = 1,2,\ldots,K \\
\sum\limits_{j = 1}^n {a_{ij}x_j \leqslant(\text{或}=,\text{或}\geqslant) b_i}     &i = 1,2,\ldots,m \\
 x_j  \geqslant 0&j = 1,2,\ldots,n \\
 d_k^ - ,\ d_k^+ \geqslant 0                                   &k = 1,2,\ldots,K \\
 \end{array} \right.}
 \end{array}
\end{equation}

模型 (\ref{eq:gp}) 的约束条件中,第一行有偏差变量,为目标约束,第二行没有偏差变量,同线性规划里的约束条件一样,
为绝对约束。可以证明,
在模型 (\ref{eq:gp}) 有解的情况下,可以将其化为只含有目标约束的目标规划问题,方法是给所有的绝对约束赋予足够
高级别的优先因子,从这个角度来看,线性规划为目标规划的特殊情况,而目标规划则为线性规划的自然推广。
根据以上分析,可将模型 (\ref{eq:gp}) 简化并用矩阵和向量记为:
\begin{equation}\label{eq:gp2}
\begin{array}{ll}
 \min \mathbf{P}(\mathbf{W}^- \mathbf{d}^- +\mathbf{W}^+ \mathbf{d}^+)\\
 \text{s.t.}{\left\{ \begin{array}{l}
 \mathbf{Ax}+ \mathbf{d}^- -\mathbf{d}^+=\mathbf{g}       \\
 \mathbf{d}^- ,\ \mathbf{d}^+ \geqslant\mathbf{0}     \\
 \mathbf{x} \geqslant \mathbf{0}                  \\
 \end{array} \right.}
 \end{array}
\end{equation}

模型 (\ref{eq:gp2}) 中,所有的约束都为目标约束,每一个目标约束都对应一对偏差变量。

\subsection{用 \pkg{goalprog} 包求解目标规划}

\R 中, \pkg{goalprog} 包\citep{goalprog08}可以求解形式为模型 (\ref{eq:gp2}) 的目标规划问题,
核心函数为 \fun{llgp()},用法如下:
\begin{verbatim}
llgp(coefficients, targets, achievements, maxiter = 1000, verbose = FALSE)
\end{verbatim}

参数中,\rcode{coefficients} 为约束变量 (不包括偏差变量) 的系数矩阵,即模型 (\ref{eq:gp2}) 中的矩阵 $\mathbf{A}$。
\rcode{targets} 为系数矩阵对应的约束向量,即模型 (\ref{eq:gp2}) 中的向量 $\mathbf{g}$。
\rcode{achievements} 为关于目标函数 (默认求最小值) 的数据框,是由 4 个向量构成:objective、priority、
p 和 n。其中数据框的每一行对应一个软约束条件,objective 和 priority 为正整数,分别表示针对第几对偏差变量
 (第 n 对偏差变量必须出现在第 n 个目标约束中) %%%%%%%%%变量 (次序和一致) 该目标约束对应的
和该偏差变量的优先级别 ,p 和 n 分别表示 $d^+$(正偏差变量)、$d^-$(负偏差变量) 的权
系数\footnote{p 和 n 分别为 positive (正的) 和 negative (负的) 的首字母,这与它们代表的偏差变量的正负是一致的。}。
\rcode{maxiter} 为最大迭代次数,取正整数,默认为 1000。
\rcode{verbose} 为逻辑变量 (取 \rcode{TRUE} 或 \rcode{FALSE} ),决定是否输出中间过程,默认不输出。
\begin{exmp}
(来自 \citep{Op07} 例题 4.2) 某工厂生产两种产品,受到原材料供应和设备工时的限制,在单位利润等有关数据已知的条件下,要求制定一个获利最大的
生产计划,具体数据见表 \ref{tab:gp001}。


 \begin{table}[ht]
   \caption{生产约束、利润表}\label{tab:gp001}
   \centering
\begin{tabular}[h]{L{3cm}|C{2cm}|C{2cm}|C{2cm}}
\hlinewd{1pt}
        产品\centering &      A &          B &     限量  \\
\hline
        原材料 (kg/件) &      5 &         10 &       60\\
%\hline
        设备工时 (h/件)&      4 &           4&       40 \\
%\hline
        利润 (元/件)   &      6 &          8 &        \\
\hlinewd{1pt}
\end{tabular}
 \end{table}
 \end{exmp}

在决策时,按重要程度的先后顺序,要考虑如下意见:
\begin{enumerate}
\item 原材料严重短缺,生产中应避免浪费,不得突破使用限额;
\item 由于产品 B 销售疲软,故希望产品 B 的产量不超过产品 A 的一半;
\item 最好能节约 4 h 的设备工时;
\item 计划利润不少于 48 元。
\end{enumerate}

以上四条意见中,显然第一条为绝对约束,第二至四条为目标约束。
请根据这些要求决定两种产品生产量。


\noindent{{\textbf  解:}这是典型的多目标规划问题,建立目标规划模型如下:}

\begin{equation*}
\begin{array}{l}
 \min \ \{P_1d_1^-,\ P_2d_2^+,\  P_3d_3^-\}\\
 \text{s.t.}\left\{ \begin{array}{rrrrrrrrr}
 5x_1&+ & 10x_2 &  &        & &        &\leqslant& 60\\
  x_1&- &   x_2 &+ &d_1^{-} &-& d_1^{+}&        =&  4\\
 4x_1&+ &  4x_2 &+ &d_2^{-} &-& d_2^{+}&        =& 56\\
 6x_1&+ &  8x_2 &+ &d_3^{-} &-& d_3^{+}&        =& 12\\
 \multicolumn{9}{l}{x_1, \ x_2,\  d_{1-3}^-, \ d_{1-3}^{+}\ \geqslant0}
 \end{array} \right.
 \end{array}
\end{equation*}

该模型中含绝对约束条件,将绝对约束条件转化为一级目标约束条件,得到模型如下:
\begin{equation*}
\begin{array}{l}
 \min \ \{P_1d_1^+,\ P_2d_1^-,\ P_3d_2^+,\  P_4d_3^-\}\\
 \text{s.t.}\left\{ \begin{array}{rrrrrrrrr}
 5x_1&+ & 10x_2 & +&d_1^{-} &-& d_1^{+}&\leqslant& 60\\
  x_1&- &   x_2 &+ &d_2^{-} &-& d_2^{+}&        =&  4\\
 4x_1&+ &  4x_2 &+ &d_3^{-} &-& d_3^{+}&        =& 56\\
 6x_1&+ &  8x_2 &+ &d_4^{-} &-& d_4^{+}&        =& 12\\
 \multicolumn{9}{l}{x_1, \ x_2,\  d_{1-4}^-, \ d_{1-4}^{+}\ \geqslant0}
 \end{array} \right.
 \end{array}
\end{equation*}

该模型符合模型 (\ref{eq:gp2}) 的形式,可以直接调用 \fun{llgp()} 函数来求解该问题,
注意:\R 中根据 \rcode{achievements} 数据框
中的 priority 来判断绝对优先级别,不用再设置 $P_1$,$P_2$,$P_3$。\R 代码和部分输出结果如下:
\begin{Verbatim}
> library(goalprog)
> coefficients=matrix(c(5,1,4,6,10,-2,4,8),4)
> targets=c(60,0,36,48)
> achievements=data.frame(objective=1:4,priority=c(1,2,3,4),
+                         p=c(1,0,1,0),n=c(0,1,0,1))
> soln=llgp(coefficients, targets, achievements)
> soln$converged
[1] TRUE
> soln$out

Decision variables
                X
X1   4.800000e+00
X2   2.400000e+00
\end{Verbatim}

观察输出结果,可以知道该问题已经得到最优解 (\verb|soln$converged| 为 \rcode{TRUE} 时,表示最优解已经找到),
此时 $x_1$,$x_2$ 分别取 4.8,2.4,即应生产 A 产品 4.8 个单位,B 产品 2.4 个单位。需要说明的是,由于约束较少,
本题有四个满意解,这里仅仅得到一个。此外,
程序结果输出较多,上面的 \verb|soln$out| 仅选取了一部分。
下面再看一个稍微复杂的例子。
\begin{exmp}
求下列目标规划问题:
 \begin{equation*}
\begin{array}{l}
 \min \ \{P_1(2d_1^+ + 3d_2^+),\ P_2d_3^-,\  P_3d_4^+\}\\
 \text{s.t.}\left\{ \begin{array}{rrrrrrrrr}
 x_1&+ & x_2 &+ &d_1^{-} &-& d_1^{+}&=& 10\\
 x_1&  &     &+ &d_2^{-} &-& d_2^{+}&=& 4\\
 5x_1&+& 3x_2&+ &d_3^{-} &-& d_3^{+}&=& 56\\
 x_1&+ & x_2 &+ &d_4^{-} &-& d_4^{+}&=&12\\
 \multicolumn{9}{l}{x_1, \ x_2,\  d_{1-4}^-, \ d_{1-4}^{+}\ \geqslant 0}
 \end{array} \right.
 \end{array}
\end{equation*}
\end{exmp}

\noindent{{\textbf  解:}这是一个多目标规划问题,可以直接调用 \fun{llgp()} 函数求解。
\R 代码及运行结果如下 (为了便于展示,输出了一些参数的信息):}
\VerbatimInput{./code/gp3.r}

分析输出结果,可以知道该问题已经得到最优解,
此时 $x_1$,$x_2$ 分别为 4,6。


通过这两个例子,我们发现该函数在解决目标规划问题上非常简捷,只需要套用格式,将问题转化为
函数需要的格式即可。而 LINGO 在求解目标规划时则比较笨拙,它需要将多目标转化为单目标,当
目标函数比较复杂时,这一过程将会非常繁琐而又困难。
